\documentclass{article}
\usepackage[utf8]{inputenc}
\usepackage{hyperref}
\usepackage[letterpaper, portrait, margin=1in]{geometry}
\usepackage{enumitem}
\usepackage{amsmath}
\usepackage{amsthm}
\usepackage{booktabs}
\usepackage{graphicx}
\usepackage{float}
\usepackage{hyperref}
\usepackage[flushleft]{threeparttable}
\usepackage{textcomp}
\usepackage{amssymb}
\usepackage{dsfont}
\hypersetup{
colorlinks=true,
    linkcolor=black,
    filecolor=black,      
    urlcolor=blue,
    citecolor=black,
}
\usepackage{natbib}
\usepackage{yhmath}
\usepackage{dutchcal}
\usepackage{titlesec}
\bibliographystyle{chicago}
\newcommand{\bib}{references.bib}
\newcommand\iid{\stackrel{\mathclap{iid}}{\sim}}
\newcommand\asym{\stackrel{\mathclap{a}}{\sim}}
\newcommand\convprob{\xrightarrow{p}}
\newcommand\convdist{\xrightarrow{d}}
\newcommand{\N}{\mathbb{N}}
\newcommand{\Z}{\mathbb{Z}}
\newcommand{\E}{\text{E}}
\newcommand{\V}{\text{Var}}
\newcommand{\Av}{\text{Avar}}
\newcommand{\se}{\text{se}}
\newcommand{\corr}{\text{Corr}}
\newcommand{\cov}{\text{Cov}}
\newcommand{\norm}{\text{Normal}}
\newcommand{\indep}{\perp \!\!\! \perp}

\begin{document}
% The tex content below is similar to the given main.tex
 
\title{Problem Set 1}
\author{Health Economics II\\
Maghfira Ramadhani\footnote{Replication codes available at \url{https://github.com/maghfiraer/health-ii-ps}}}
\date{\today}
\maketitle

\section*{1 Logit Demand}

We are given data on over-the-counter (OTC) headache medicine. The data is at the store-week level for four brands and three package sizes. A brand-size pair is a product.
Consider the utility function for product $j$ in market $t$ for consumer $i$:
\begin{align*}
    u_{ijt} &=  X_{jt} \beta  - \alpha p_{jt}+ \xi_{jt} + \varepsilon_{ijt} \\
    &=\delta_{jt}+\varepsilon_{ijt}.
\end{align*}
where $\varepsilon_{ijt}$ is an i.i.d. draw from type I extreme value distribution, $X_{jt}$ is a vector of product characteristics, $p_{jt}$ is the price of product $j$ in market $t$, and $\xi_{jt}$ is an unobserved characteristic of the product.
\begin{enumerate}
\item Table \ref{t1:summary} shows summary statistics that provides the mean of each of the following for each brand-
size pair: market share of sales, unit price, price/50 tab, and wholesale price.
\begin{table}[H]
\centering
\begin{threeparttable}
\caption{Summary Statistics}\label{t1:summary}
{
\def\sym#1{\ifmmode^{#1}\else\(^{#1}\)\fi}
\begin{tabular}{l*{11}{c}}
\hline\hline
                    &\multicolumn{3}{c}{Advil}             &\multicolumn{3}{c}{Bayer}             &\multicolumn{2}{c}{Store Brand}&\multicolumn{3}{c}{Tylenol}           \\\cmidrule(lr){2-4}\cmidrule(lr){5-7}\cmidrule(lr){8-9}\cmidrule(lr){10-12}
                    &\multicolumn{1}{c}{25}&\multicolumn{1}{c}{50}&\multicolumn{1}{c}{100}&\multicolumn{1}{c}{25}&\multicolumn{1}{c}{50}&\multicolumn{1}{c}{100}&\multicolumn{1}{c}{50}&\multicolumn{1}{c}{100}&\multicolumn{1}{c}{25}&\multicolumn{1}{c}{50}&\multicolumn{1}{c}{100}\\
\hline
Market share of sales&       0.120&       0.077&       0.036&       0.043&       0.035&       0.082&       0.093&       0.071&       0.149&       0.178&       0.116\\
Unit price          &       0.119&       0.103&       0.082&       0.107&       0.072&       0.040&       0.039&       0.044&       0.137&       0.099&       0.070\\
Price per 50 tab    &       5.927&       5.145&       4.080&       5.346&       3.607&       1.983&       1.928&       2.224&       6.841&       4.942&       3.508\\
Wholesale price     &       2.030&       3.623&       6.091&       1.847&       2.422&       3.712&       0.910&       1.919&       2.182&       3.672&       5.755\\
\hline\hline
\end{tabular}
}

\end{threeparttable}
\end{table}

If we assume that $\varepsilon_{ijt}$ is distributed type I extreme value which have the density $f$ and cumulative distribution $F$ as follows
\begin{align*}
    f(\varepsilon_{ijt})&=e^{-\varepsilon_{ijt}}e^{e^{-\varepsilon_{ijt}}}\\
    F(\varepsilon_{ijt})&=e^{-e^{-\varepsilon_{ijt}}}
\end{align*}

Consequently, the probability that consumer $i$ chooses product $j$ in market $t$ is given by
\begin{align}
    \mathcal{s}_{jt}&=\Pr(\delta_{jt}+\varepsilon_{ijt}>\delta_{kt}+\varepsilon_{ikt} \quad \forall k\neq j) \notag\\
    \Leftrightarrow\mathcal{s}_{jt}&=\Pr(\varepsilon_{ikt}<\varepsilon_{ijt}+\delta_{jt}-\delta_{kt} \quad \forall k\neq j) \notag\\
    \Leftrightarrow\mathcal{s}_{jt}&=\int \left(\prod_{k\neq j} F(\varepsilon_{ijt}+\delta_{jt}-\delta_{kt})\right)f(\varepsilon_{ijt})d\varepsilon_{ijt} \notag\\
    \Leftrightarrow\mathcal{s}_{jt}&=\frac{\exp(\delta_{jt})}{\sum_{k}\exp(\delta_{kt})}.\label{e1:ps1}
\end{align}
If we assume that the mean utility of the outside good is zero, then the market share of sales for product $j$ in market $t$ is given by
\begin{align}
    \mathcal{s}_{0t}&=\frac{1}{\sum_{k}\exp(\delta_{kt})}.\label{e2:ps1}
\end{align}
Dividing equation \eqref{e1:ps1} by equation \eqref{e2:ps1} yields
\begin{align}
    &\frac{\mathcal{s}_{jt}}{\mathcal{s}_{0t}}=\exp(\delta_{jt})\notag\\
    &\Leftrightarrow \ln(\mathcal{s}_{jt})-\ln(\mathcal{s}_{0t})= X_{jt} \beta  - \alpha p_{jt}+ \xi_{jt}\notag \\
    &\Leftrightarrow \ln(\mathcal{s}_{jt})=\ln(\mathcal{s}_{0t}) + X_{jt} \beta  - \alpha p_{jt}+ \xi_{jt}.\label{e3:ps1} 
\end{align}
We can estimate equation \eqref{e3:ps1} with log market share of sales as the outcome variable and price and product characteristics as the explanatory variables using OLS, or IV to solve the price endogeneity problem. 
\item The demand model estimates by OLS using price and promotion as product characteristics is shown in column (1) of Table \ref{t2:OLS}.
\item The demand model estimates by OLS using price and promotion as product characteristics including product dummies is shown in column (2) of Table \ref{t2:OLS}.
\item The demand model estimates by OLS using price and promotion as product characteristics, including store-product (the interaction of store and product) dummies is shown in column (3) of Table \ref{t2:OLS}.

\begin{table}[H]
\centering
\begin{threeparttable}
\caption{Demand model OLS estimates}\label{t2:OLS}
\begin{tabular}{l*{3}{c}}
\hline\hline
                    &\multicolumn{3}{c}{OLS}               \\\cmidrule(lr){2-4}
                    &\multicolumn{1}{c}{(1)}&\multicolumn{1}{c}{(2)}&\multicolumn{1}{c}{(3)}\\
\hline
Price               &      -0.049&      -0.277&      -0.297\\
                    &     (0.002)&     (0.009)&     (0.009)\\
[1em]
Promotions          &       0.154&       0.287&       0.277\\
                    &     (0.016)&     (0.012)&     (0.011)\\
\hline
Fixed Effect        &           -&     Product&Store-Product\\
Adjusted-R$^2$      &       0.014&       0.048&       0.036\\
Observations        &       38544&       38544&       38544\\
\hline\hline
\end{tabular}

\end{threeparttable}
\end{table}

\item The demand estimates using wholesale price as an instrument is shown in column (1) - (3) of Table \ref{t3:IV}.
\item The demand estimates using Hausman instrument (i.e. the average price of the same product in other
stores) is shown in column (4) - (6) of Table \ref{t3:IV}.

\begin{table}[H]
\centering
\begin{threeparttable}
\caption{Demand model IV estimates}\label{t3:IV}
\begin{tabular}{l*{6}{c}}
\hline\hline
                    &\multicolumn{3}{c}{IV: Wholesale Price}&\multicolumn{3}{c}{IV: Hausman Instrument}\\\cmidrule(lr){2-4}\cmidrule(lr){5-7}
                    &\multicolumn{1}{c}{(1)}&\multicolumn{1}{c}{(2)}&\multicolumn{1}{c}{(3)}&\multicolumn{1}{c}{(4)}&\multicolumn{1}{c}{(5)}&\multicolumn{1}{c}{(6)}\\
\hline
Price               &      -0.010&       0.016&       0.008&      -0.051&      -0.501&      -0.501\\
                    &     (0.003)&     (0.018)&     (0.018)&     (0.002)&     (0.012)&     (0.012)\\
[1em]
Promotions          &       0.176&       0.376&       0.371&       0.154&       0.219&       0.215\\
                    &     (0.016)&     (0.013)&     (0.012)&     (0.016)&     (0.012)&     (0.011)\\
\hline
Fixed Effect        &           -&     Product&Store-Product&           -&     Product&Store-Product\\
Montiel-Pflueger F-statistics&      276954&       14225&       14970&     2498670&       50394&       64374\\
Adjusted-R$^2$      &       0.008&       0.024&       0.008&       0.014&       0.034&       0.023\\
Observations        &       38544&       38544&       38544&       38544&       38544&       38544\\
\hline\hline
\end{tabular}

\end{threeparttable}
\end{table}

\item The IV estimates when instrumenting for price using wholesale price is shown in column (1) - (3) of Table \ref{t3:IV}. The coefficient for prices are practically zero. This is because the wholesale price is not a good instrument for price. I suspect that wholesale price is highly correlated with product price but fail to satisfy the exclusion restriction, this could happen when the store collect the same portion of markup from the price of product sold. The result from using Hausman instrument in column (1) - (3) of Table \ref{t3:IV} show that it is a better instrument for price. The specification in column (6) is the best model in addressing endogeneity issues since it control for unobserved store-product characteristics as well as using a good instrument for price.

\item The own-price elasticities for products $j$ in the market $t$ is given by
\begin{align*}
    \frac{\partial\mathcal{s}_{jt}}{\partial p_{jt}}\frac{p_{jt}}{\mathcal{s}_{jt}}=-\alpha p_{jt}(1-\mathcal{s}_{jt}).
\end{align*}
From the data, the mean own-price elasticity for each product $j$ is given by
\begin{align*}
    \left.\frac{\partial\mathcal{s}_{jt}}{\partial p_{jt}}\frac{p_{jt}}{\mathcal{s}_{jt}}\right\vert_t=-\alpha \frac{1}{T}\sum_{t}p_{jt}(1-\mathcal{s}_{jt}).
\end{align*}
The mean own-price elasticity for each product $j$ is shown in Table \ref{t4:elasticity}.

\begin{table}[H]
\centering
\caption{Mean own price elasticity for different product across specifications}\label{t4:elasticity}
\begin{threeparttable}
{
\def\sym#1{\ifmmode^{#1}\else\(^{#1}\)\fi}
\begin{tabular}{l*{11}{c}}
\hline\hline
                    &\multicolumn{3}{c}{Advil}             &\multicolumn{3}{c}{Bayer}             &\multicolumn{2}{c}{Store Brand}&\multicolumn{3}{c}{Tylenol}           \\\cmidrule(lr){2-4}\cmidrule(lr){5-7}\cmidrule(lr){8-9}\cmidrule(lr){10-12}
                    &\multicolumn{1}{c}{25}&\multicolumn{1}{c}{50}&\multicolumn{1}{c}{100}&\multicolumn{1}{c}{25}&\multicolumn{1}{c}{50}&\multicolumn{1}{c}{100}&\multicolumn{1}{c}{50}&\multicolumn{1}{c}{100}&\multicolumn{1}{c}{25}&\multicolumn{1}{c}{50}&\multicolumn{1}{c}{100}\\
\hline
Specification (1)   &      -0.027&      -0.050&      -0.082&      -0.027&      -0.036&      -0.038&      -0.018&      -0.043&      -0.030&      -0.042&      -0.065\\
Specification (2)   &       0.042&       0.076&       0.126&       0.041&       0.056&       0.059&       0.028&       0.067&       0.047&       0.065&       0.100\\
Specification (3)   &       0.020&       0.037&       0.061&       0.020&       0.027&       0.028&       0.014&       0.032&       0.023&       0.032&       0.048\\
Specification (4)   &      -0.132&      -0.240&      -0.398&      -0.129&      -0.176&      -0.184&      -0.089&      -0.209&      -0.147&      -0.206&      -0.313\\
Specification (5)   &      -1.306&      -2.381&      -3.939&      -1.280&      -1.742&      -1.824&      -0.879&      -2.073&      -1.458&      -2.037&      -3.106\\
Specification (6)   &      -1.307&      -2.382&      -3.941&      -1.281&      -1.743&      -1.825&      -0.879&      -2.074&      -1.459&      -2.037&      -3.107\\
\hline\hline
\end{tabular}
}

\end{threeparttable}
\end{table}

\item Let be $\mathcal{J}$ the choice sets of available products. Under the assumptions of the logit model, the expected consumer surplus that an individuals receives
from getting to choose the OTC headache medicine that maximizes their utility is
\begin{align*}
\E[CS_i|j\in \mathcal{J}]&=\frac{1}{\alpha} \E[\max_j (\delta_{jt}+\varepsilon_{ijt})]\\
&=\frac{1}{\alpha} \ln\left(\sum_{j\in \mathcal{J}}\exp(\delta_{jt})\right)
\end{align*}

When a product $l$ is not offered in the market, the expected consumer surplus that an individuals receives from getting to choose the OTC headache medicine that maximizes their utility is
\begin{align*}
    \E[CS_i|j\in \mathcal{J}\diagdown l]&=\frac{1}{\alpha} \ln\left(\sum_{j\in \mathcal{J}\diagdown l}\exp(\delta_{jt})\right)\\
    &=\frac{1}{\alpha} \ln\left(\sum_{j\in \mathcal{J}}\exp(\delta_{jt})-\exp{(\delta_{lt})}\right)
\end{align*}

The change in consumer surplus from removing product $l$ from the choice set is given by
\begin{align*}
    \E[\Delta CS_i]&=\frac{1}{\alpha} \ln\left(\frac{\sum_{j\in \mathcal{J}}\exp(\delta_{jt})-\exp{(\delta_{lt})}}{\sum_{j\in \mathcal{J}}\exp{(\delta_{jt})}}\right)\\
    &=\frac{1}{\alpha}\ln\left(1-\frac{\exp{(\delta_{lt})}}{\sum_{j\in \mathcal{J}}\exp{(\delta_{jt})}}\right)\\
    &=\frac{1}{\alpha}\ln\left(1-\mathcal{s}_{lt}\right)
\end{align*}
Using the data, the mean change in consumer surplus from removing product $l$ from the choice set is given by
\begin{align*}
    \Delta CS_i\vert_t&=\frac{1}{\alpha} \frac{1}{T}\sum_{t}\ln\left(1-\mathcal{s}_{lt}\right)
\end{align*}

The mean change in consumer surplus from removing a specific product from a market is shown in Table \ref{t5:surplus}.

\begin{table}[H]
\centering
\caption{Mean change in consumer surplus from removing specific product}\label{t5:surplus}
\begin{threeparttable}
{
\def\sym#1{\ifmmode^{#1}\else\(^{#1}\)\fi}
\begin{tabular}{l*{11}{c}}
\hline\hline
                    &\multicolumn{3}{c}{Advil}             &\multicolumn{3}{c}{Bayer}             &\multicolumn{2}{c}{Store Brand}&\multicolumn{3}{c}{Tylenol}           \\\cmidrule(lr){2-4}\cmidrule(lr){5-7}\cmidrule(lr){8-9}\cmidrule(lr){10-12}
                    &\multicolumn{1}{c}{25}&\multicolumn{1}{c}{50}&\multicolumn{1}{c}{100}&\multicolumn{1}{c}{25}&\multicolumn{1}{c}{50}&\multicolumn{1}{c}{100}&\multicolumn{1}{c}{50}&\multicolumn{1}{c}{100}&\multicolumn{1}{c}{25}&\multicolumn{1}{c}{50}&\multicolumn{1}{c}{100}\\
\hline
Specification (1)   &     -12.391&      -7.840&      -3.549&      -4.269&      -3.413&      -8.255&      -9.557&      -7.299&     -15.615&     -18.977&     -12.001\\
Specification (2)   &       8.046&       5.091&       2.305&       2.772&       2.216&       5.360&       6.206&       4.739&      10.139&      12.323&       7.793\\
Specification (3)   &      16.686&      10.558&       4.779&       5.749&       4.596&      11.116&      12.869&       9.829&      21.027&      25.555&      16.161\\
Specification (4)   &      -2.558&      -1.619&      -0.733&      -0.881&      -0.705&      -1.704&      -1.973&      -1.507&      -3.224&      -3.918&      -2.478\\
Specification (5)   &      -0.258&      -0.163&      -0.074&      -0.089&      -0.071&      -0.172&      -0.199&      -0.152&      -0.325&      -0.395&      -0.250\\
Specification (6)   &      -0.258&      -0.163&      -0.074&      -0.089&      -0.071&      -0.172&      -0.199&      -0.152&      -0.325&      -0.395&      -0.250\\
\hline\hline
\end{tabular}
}

\end{threeparttable}
\end{table}

\end{enumerate}

\section*{2 Adverse selection and (selection on) moral hazard}

Define each of the following terms and describe the data and identifying variation necessary to estimate their effects in health insurance markets: adverse selection, moral hazard, and selection on moral hazard. What approaches have the papers discussed in class taken to estimate these effects? What have these papers found and concluded?

\subsection*{Adverse selection}
Adverse selection happens when willingness to pay is positively correlated. The source of asymetric information because the insurance company do not observe the individual health status. Under this condition, individual who have the highest willingness to pay are individual with the highest expected cost and consequently the average cost is always higher than the marginal cost causing underinsurance. 

\cite{einav2010estimating} use the data on health insurance options, choices, and medical insurance claims of Alcoa, Inc in 2004 in which they focused on subsample of 3,779 employes who chose one of the two modal insurance plans, contracts $H$ and $L$. They use price variation in which due to Alcoa's organization structure, employees doing similar jobs are exposed to different the price of insurance for the same sets of coverage options across different section of the company. Using this variation, they estimate the marginal cost and demand curve, and find that the marginal cost is increasing in price (decreasing in quantity) which is consistent with adverse selection.

\subsection*{Moral hazard}
Under moral hazard, individual who choose more comprehensive coverage have less incentive to reduce claim probability, i.e. being more risky. The source of asymetric information is due to the change in behaviour of the individual after the insurance is purchased. 

\cite{einav2010estimating} use the same data and price variation to estimate the effect of moral hazard by computing marginal cost curve for contracts $H$ and $L$. They define moral hazard as the difference in the expected cost of the individual who choose the high coverage and the individual who choose the low coverage. They find that they can not reject the null of no moral hazard.

\cite{brot2017does} use the data from the RAND Health Insurance Experiment (HIE) in which family are randomly enrolled in either free healthcare or healtcare with some cost-sharing up to an OOP maximum. They use the variation in monthly spending between those with and without cost-sharing to estimate the effect of moral hazard. They find that individual respond to both current price and expected future price when choosing care, i.e. individual who already passed their deductible at the end of the year will consume more care compared to the very firs month of the year after.

\subsection*{Selection on moral hazard}
Selection on moral hazard is when individual selecting insurance coverage based on their underlying health risk or status and their behavior response after coverage.

\cite{einav2013selection} use the data from Alcoa (the same company as in \cite{einav2010estimating}) from 2003 and 2004, in some analysis the data is extended through 2006. The data concain the menu of health insurance options available to each employee, the employee's coverage choices, and detailed claim-level information of the employee (and any covered dependents') utilization and expenditure for the year. It also comtaom information on demographics of employee, union affiliation, whether theey are hourly or salary type employee, age, race, gender, annual earnings, job tenure, number and ages of insuren family members. To identify and estimate moral hazard they use variation in the health insurance options offered to different group of workers at different points of time in the imposition of the new PPO plans. They find significant heterogeneity in moral hazard and selection on moral hazard. Individual who has a higher behaviour response to coverage are more likely to select the more generous coverage.

\bibliography{\bib}

\end{document}